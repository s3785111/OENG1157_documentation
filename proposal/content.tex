\section{Summary}
\section{Problem statement}
The Data Analytics and Verification team is to operate as a subset within the AURC Avionics Team for the HIVE Aurora V rocket project. This subsystem's primary objective is to design, test, and validate data analytics processes for the avionics, ground communications, redundant electronic systems, and payload systems.  

The project scope spans across five iterations of the Aurora rocket (A1 through to A5). As each rocket iteration progresses, data analysis processes should change to improve the accuracy of verification processes. Feedback during analysis is also to be provided to the broader Avionics team.     

\subsection{Project Deliverables}
Data analytics and verification processes will include:  
\begin{itemize}
  \item Real-Time Processing: Research and develop algorithms for sensor fusion to integrate data from multiple sensors. Implement filtering techniques to remove noise. 
  \item Post-Processing: Perform statistical analysis to evaluate key performance metrics of avionics systems. Implement systems to detect deviations from expected behaviour when comparing raw sensor and fused data to the data captured by the blue raven.     
  \item Data Visualization: Develop tools for generating informative graphs and charts for data interpretation. Explore the possibility of creating simulations based on collected data. 
  \item Data Validation: Determine the validity of the captured data. Identify the nature of errata (error with our analysis methods or hardware issue?) 
\end{itemize}


\subsection{Tools to Undertake Data Analytics and Verification}
\begin{itemize}
  \item Github  
  \item Arduino IDE, Visual Studio IDE  
  \item Arduino libraries  
  \item Python libraries  
\end{itemize}

\section{Background and literature review}
The Aurora V project aims to build a L3 rocket designed to fly with apogee at exactly 30,000ft. To achieve this goal, accurate data collection and analysis is a necessity both to understand the flight path of the rocket and its contributing factors to facilitate correcting modifications to the rocket itself, as well as to supply real-time state information to the airbrake system for dynamic adjustments to drag coefficients.

Inertial sensors are susceptible to noise in their measurements, which introduces error into captured data and accumulates a drift in time-varying calculations. Many model rockets such as Pioneer Rocketry's Skybreaker\cite{pioneer-rocketry} implement some variation of a Kalman Filter to smooth out this noise for more accurate state estimation. A Kalman Filter attempts to provide an accurate estimation for noisy processes by using a model in combination with previous estimates to predict the future state of the process, correcting this prediction with measured data\cite{kalman-introduction,kalman1960}. This process accounts for the process noise - deviation between the true state of the process and the state as described by the model - as well as the noise generated by the measurement of state values. A Kalman gain is computed to weight how much of the modeled and measured data should be accounted for in the new estimate, with an error covariance matrix tracking the uncertainty in the estimate.

Rockets built by university teams such as uORocketry and NTNU generally make use of positional data such as height or displacement together with velocity measurements to control their airbrake systems\cite{uORocketry, NTNU}. To accurately determine these state variables, sensor fusion techniques may be employed to reduce the accumulated error that results from the measurement process. 

An example of how this may be applicable is in the altitude calculations. The most simple method of determining altitude would be to directly use the barometric pressure, however this is highly dependant on calibration at the launch site and susceptible to noise introduced from the sensor. An alternative would be to use integrated accelerometer data to obtain position, however the process of integrating itself is prone to drift due to time lag as well as flight angle offsets. A potential solution for these problems would be to combine the barometric altitude estimate together with positional calculations from the integrated accelerometer data, as described in a 2004 report by David Schulz\cite{kalman-apogee}. Accelerometer data can also be combined together with gyroscopic measurements from the IMU to rotate the inertial frame, allowing for pure vertical acceleration data to be integrated down to positional information. In all cases, some application of a Kalman filter appears necessary.


\section{Research questions}
\begin{questions}
  \item What variables are necessary to track in the rocket state?
  \item What algorithms should be run to accurately estimate rocket state in real time?
  \item How will sensor fusion techniques be applied to obtain state information from available data?
  \item What statistical methods will be employed to assess real time validity of data? \begin{questions}
    \item What communication protocols will be used to inform the redundant systems of detected errata?
    \item What metric(s) will the redundant systems use to elect which of primary and secondary sensors to sample?
  \end{questions}
  \item What visualisation methods will be applied to provide state representation in real time?
  \item What visualisation methods will be applied to provide state information offline? \begin{questions}
    \item What techniques will be applied to analyse trends in offline data? 
    \item What trends are important to analyse in offline data?
  \end{questions}
\end{questions}

\section{Methodology}
The high-level flow chart shown in Figure~\ref{fig:flowchart-high_level} depicts how raw sensor data is to be captured and transmitted from a firmware perspective.  The purpose of the flow chart is to guide avionics on how data is to be logged and transmitted to enable post flight analysis and verification. The system is primarily broken down into a high-resolution interval at 500Hz and low-resolution interval at 50Hz. These resolution intervals are intended to match with Blue Raven intervals in order to sync and compare data sets post flights. The Blue Raven data will act as benchmark to determine the accuracy of data collected from various sensors of the avionics unit. Logging of data will be triggered based on a launch event recorded from the Blue Raven to enable both systems to collect data simultaneously. 

During flight, captured sensor data will be written to an onboard flash memory and transmitted over LoRa to ground communications, providing a level of redundancy to mitigate the risk of data loss. This dual approach ensures that if one method encounters failure or interruption, data can still be processed and analyzed post flight. While flash memory can only be analyzed post flight, LoRa communications provides critical live communications to the ground, enabling real-time monitoring and analysis of sensors and microcontrollers. Real-time processing through the use of algorithms is to be developed with the purpose of detecting hardware failures and or abnormalities to inform avionic systems to switch to redundant sensors or microcontrollers if required.  

\begin{figure}
  \begin{center}
    \includegraphics[width=0.95\textwidth]{img/flowchart-high_level.png}
  \end{center}
  \caption{High-level functional flowchart of data sampling and logging}\label{fig:flowchart-high_level}
\end{figure}

Only Raw data will be collected and transmitted from each sensor register during flight. The aim of this approach is to optimize data collection and transmission rather than processing which will be conducted post flight. By doing so, onboard processing will be minimized, reducing the risk of processing errors and ensuring efficient utilization of LoRa bandwidth and onboard flash memory. 

During the high-resolution interval, the system reads data from an IMU (Inertial Measurement Unit) which consists of an accelerometer, gyroscope, and magnetometer sensors. These sensors are essential in determining the aircraft motion and orientation; therefore, a higher sampling rate is required to capture subtle changes accurately. Raw data from sensor registers are then combined with header to form a data frame which will be stored in a buffer array ready to be stored in flash. Sensor data will additionally be kept in RAM to be transmitted during the low-resolution interval.  At the low-resolution interval, the system reads data from a barometer, which includes a pressure and temperature sensor to determine altitude. Similar to the high-resolution interval, raw data are used to form a data frame which is added to a buffer array serving as temporary storage before the data is written to flash. Once pressure and temperature data frame is formed and saved to RAM, the buffer array is written to flash and data stored in RAM will be transmitted over LoRa. Switch I/O will also be written to RAM and transmitted to ground control to notify avionic systems if a switch has turned off.  

A number of alternate design methods were considered which included a separate logging interval where transmission and saving to flash occurred over a separate time frame. Although having a different interval provides greater control, flash writing and LoRa transmission occurring during on the low-resolution interval simplifies the design, consolidating data handling tasks.  

\subsection{Mid-flight processing}

\section{Risk assessment}
From an occupational health and safety perspective, the risks associated with this project are minimal. Due to the nature of this project, most risks will be office related, which include, visual strain, neck, and back pain from long hours in front of computers causing eye strain, headaches, and fatigue. Although these risks pose health concerns, they can be effectively managed through ergonomic interventions such as the use of ergonomic furniture and regular breaks during project work. Additionally, some aspects of testing verification will include interaction with hardware such as sensors and other electronic modules. 

While these interactions may introduce physical hazards such as the risk of electrical shocks or injury from handling equipment, hardware is to be handled in RMIT Laboratory rooms where proper procedures and PPE will be utilized. Hardware will also be regularly inspected and maintenance to help identify and address potential safety hazards proactively.  

As data analysis and verification is heavily reliant on the avionic hardware, there are foreseeable risks that may impact how data analysis and verification is approached. The figure below outlines possible risks and contingency strategies to ensure project deliverables are met.  

\begin{table}[]
\centering
\begin{tabular}{>{\raggedright}p{4cm}p{1.5cm}p{1.5cm}p{7cm}}
\toprule
\textbf{Risk} & \textbf{likelihood} & \textbf{Impact} & \textbf{Contingency Strategy} \\ \midrule
Delay arrival of avionic hardware & High & High & \vspace{-1.5em}
\begin{itemize}[leftmargin=*]
  \item Work closely with Avionics team to understand if there are any Hardware changes or substitutions.
  \item Prioritise non hardware dependent tasks.
  \item Focus on preliminary data analysis, algorithm development, or simulation testing while awaiting hardware arrival.
\end{itemize}\\\midrule
Data collected from sensors is corrupted or incomplete & Medium & High & \vspace{-1.5em}
\begin{itemize}[leftmargin=*]
  \item Develop an algorithm for data validation and error detection to identify and filter out unreliable data.
\end{itemize}\\\midrule
Loss of ground communication during flight & Medium & High & \vspace{-1.5em}
\begin{itemize}[leftmargin=*]
  \item Testing processes to ensure that data can be extracted from flash.
\end{itemize}\\\midrule
Flash data unrecoverable & Medium & High & \vspace{-1.5em}
\begin{itemize}[leftmargin=*]
  \item Ensure that enough senor data is being transmitted in both high- and low-resolution transmissions. This will still enable data analysis to take place even if flash data is lost.
\end{itemize}\\\midrule
Scope creep & Low & Medium & \vspace{-1.5em}
\begin{itemize}[leftmargin=*]
  \item Establish clear project scope.  
  \item Prioritise data analysis and verification task.  
  \item Regularly review project scope and assess proposed changes with Glenn Matthews.
\end{itemize}\\\midrule
Integration Challenges with ground communication and real-time processing & Medium & Medium & \vspace{-1.5em}
\begin{itemize}[leftmargin=*]
  \item Understand the specifications between the communication hardware and processing algorithm. 
  \item Conduct integration checks and testing.
\end{itemize}\\
\bottomrule
\end{tabular}
\end{table}
