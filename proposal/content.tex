\section{Summary}
\section{Problem statement}
\section{Background and literature review}
The Aurora V project aims to build a L3 rocket designed to fly with an apogee at exactly 30,000ft. To achieve this goal, accurate data collection and analysis is a necessity both to understand the flight path of the rocket and its contributing factors to facilitate correcting modifications to the rocket itself, as well as to supply real-time state information to the airbrake system for dynamic adjustments to drag coefficients.

Inertial sensors are susceptible to noise in their measurements, which introduces error into captured data and accumulates a drift in time-varying calculations. Many model rockets such as Pioneer Rocketry's Skybreaker\cite{pioneer-rocketry} implement some variation of a Kalman Filter to smooth out this noise for more accurate state estimation. A Kalman Filter attempts to provide an accurate estimation for noisy processes by using a model in combination with previous estimates to predict the future state of the process, correcting this prediction with measured data\cite{kalman-introduction,kalman1960}. This process accounts for the process noise - deviation between the true state of the process and the state as described by the model - as well as the noise generated by the measurement of state values. A Kalman gain is computed to weight how much of the modeled and measured data should be accounted for in the new estimate, with an error covariance matrix tracking the uncertainty in the estimate.

Rockets built by university teams such as uORocketry and NTNU generally make use of positional data such as height or displacement together with velocity measurements to control their airbrake systems\cite{uORocketry, NTNU}. To accurately determine these state variables, sensor fusion techniques may be employed to reduce the accumulated error that results from the measurement process. 

An example of how this may be applicable is in the altitude calculations. The most simple method of determining altitude would be to directly use the barometric pressure, however this is highly dependant on calibration at the launch site and susceptible to noise introduced from the sensor. An alternative would be to use integrated accelerometer data to obtain position, however the process of integrating itself is prone to drift due to time lag as well as flight angle offsets. A potential solution for these problems would be to combine the barometric altitude estimate together with positional calculations from the integrated accelerometer data, as described in a 2004 report by David Schulz\cite{kalman-apogee}. Accelerometer data can also be combined together with gyroscopic measurements from the IMU to rotate the inertial frame, allowing for pure vertical acceleration data to be integrated down to positional information. In all cases, some application of a Kalman filter appears to be necessary.

\section{Research questions}
\section{Methodology}
\section{Risk assessment}


